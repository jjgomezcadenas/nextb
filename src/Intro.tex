\section{Introduction}\label{sec:intro}

Neutrinoless double beta decay (\bbonu) is a postulated very slow radioactive process in which two neutrons inside a nucleus transform into two protons emitting two electrons. The discovery of this process would demonstrate that neutrinos are Majorana particles and that total lepton number is not conserved in nature, two findings with far-reaching implications in particle physics and cosmology \cite{Gomez_2013,GomezCadenas:2013ue,Cadenas_2012}.

After 75 years of experimental effort, no compelling evidence for the existence of \bbonu\ decay has been obtained. If a discovery is not made by the current generation of \bbonu\ searches (see section \ref{sec:experiments} new experiments must be designed capable of simultaneously increasing the fiducial masses up to the ton scale, while reducing the backgrounds in the ROI by one to two orders of magnitude. A full exploration of the inverse hierarchy implies reaching sensitivities of 20 meV in the Majorana mass (section \ref{sec:boonu}). In turn, reducing the limit in \mbb\ by a factor five, requires reducing the limit in the \bbonu\ decay by a factor 25. This is a tremendous experimental challenge which requires to deploy masses in the ton scale, while reducing the backgrounds in the ROI to about {\em one or less} counts per ton of isotope mass and year. 

In this paper we show that a Magnetized Gaseous Experiment (MAGE) can, indeed reach a background rate of less than one count per ton of isotope and year of exposure. We discuss two different approaches. A magnetized High Pressure Xenon TPC, that we will call BEXT\footnote{Proposed by one of us (Gomez-Cadenas) an evolution of the NEXT experiment.}, and a magnetized ion Selenium TPC (MIST)\footnote{A novel concept, proposed by one of us (Nygren).} While BEXT is based in today's technology, MIST may require significant R\&D before it can be realized. However, if feasible, it would use of an important isotope (\SE) which at present is only being studied by the Super-NEMO demonstrator, which in turn uses a technology that cannot be extrapolated to large masses (e.g, ton scale). 

This paper is organized as follows. In section \ref{sec.bbonu} we briefly review the phenomenology of neutrinoless double beta decay. In ref{sec.bb} we summarize the state-of-the-art in \bbonu\ searches and show that background rates of less than one count per ton and year of exposure are needed in order to fully explore the so-called inverse hierarchy of neutrino masses with exposures in the range of the ton-year. In section \ref{sec:bext} we describe a conceptual design for BEXT. Section \ref{sec:MIST} presents MIST. In section 
\ref{sec.trk} we discuss tracking of electrons in a dense gas and show that a sufficiently intense magnetic field (in the range of 0.5 to 1 Tesla) combined with reasonably good resolution in the measurement of the track trajectory permits the determination of the {\em average sign of the electron(s)}, which can be used to discriminate between signal and background events. In section \ref{sec.results} we present the results of our Monte Carlo simulations.

