\section{MIST: A Magnetic Ion Selenium TPC}
\subsection{SeF$_6$~ as an alternative to \XE}

\SE\ has a \Qbb\ of 2995 keV and a natural abundance of about 9.2\%, making it a very attractive \bbonu\ candidate.  \SE\ has been studied by the NEMO experiment and is the isotope of choice for the Super-NEMO demonstrator, where a thin target sheet made of \SE\ is surrounded by a tracker and a calorimeter, immersed in a weak magnetic field. Alas, the Super-NEMO technology cannot be extrapolated to the ton scale. Alternatively consider selenium hexafluoride, SeF$_6$. This is a gas at room temperature up to nearly 10 bars, and therefore one can imagine a high-pressure SeF$_6$~ chamber. Given that there is one atom of \SE\ for each 6 atoms of Fluorine, the \SE\ isotope mass corresponds to about half of the SeF$_6$~mass. On the other hand, the average $Z$~ of the SeF$_6$~molecule is $(6 \times 9 + 34)/7 \sim12$, a factor 4.5 smaller than the $Z$~of xenon. Moliere  theory establishes that multiple scattering is proportional to Z, thus one can expect four times less multiple scattering in SeF$_6$~than in xenon and about 2 times less error associated to coordinate measurement (since measurement error goes with the square root of $Z$).  SeF$_6$~ is, however, highly electronegative, and, unfortunately, rather toxic, a feature that will, surely, complicate the feasibility of the scenario considered here.

\subsection{Detection: Ionization Imaging}

A TPC is essential to provide the track quality needed. However, due to its high electronegativity, SeF$_6$~makes impractical any attempt to realize well-behaved avalanche gain. In the absence of free electrons, no electroluminescence can be expected either.  In the presence of a substantial external electric field, free electrons might have a mean survival time sufficient that the cation-electron track images separate quickly on a scale corresponding to Onsager capture radius, before the symmetric pair of cation-anion images are formed. In the absence of recombination (whose existence or not must be determined experimentally), we assume, for this discussion, that recombination can be neglected.  

Any source of ionization, including \bb\ decay, thus leads to two track images with equivalent information moving in opposite directions: one cation track and one anion track. Detection of both track images can improve energy resolution, perhaps by $1\sqrt(2)$.  In any case, though, detection of both images is essential for placement of the event within the TPC active volume.   

With the requirement of 3-D imaging, and in the absence of avalanche gain, one is left to consider ultra-low-noise direct detection of ionization as proposed in REF (In 2007 I published a paper, "Ionization Imaging–A New Way to Search for 0ν-ββ ") based on pixelized detection of charge in a gaseous TPC.  The idea is to arrange the anode plane as a set of macroscopic pixels to collect charge with high efficiency, but without any avalanche gain. One condition to be met is to size the pixel sufficiently small that nearly all track topologies are captured with sufficient accuracy and clarity, but not so small that signal/noise is unfavorable.  Even if S/N were not a problem, diffusion places a lower limit on the physical size of pixels. For the ions of interest here, thermal rms diffusion after a meter drift is on the order of 1 mm, so a pixel size of 5 mm would be sufficient. This results in about 280 k pixels in each readout plane for a cylindrical fiducial volume of $R=1.5$~m (see discussion in previous section). We assume that energy resolution should not be worst than what can be achieved in a HPXe TPC.   

Typical ionic drift velocity is around 1 mm/ms, suggesting the need for sampling at 0.5 ms to capture the profile of arriving charge. However, CMOS image/charge sensing has a noise sweet spot in shaping time in the range of 10 - 100 $\mu$s.  Since only single channel optimization is relevant here, it may be possible to stretch shaping out to 500 $\mu$s without a very large increase in noise.  The power/real estate constraints imposed in multi-pixel circuitry can be easily relaxed, and a process that includes JFETs should help. About six samples in time would be needed for tracks lying mainly in the x-y plane. With six samples in time for each of 60 --100 spatial samples, the total number of samples would be around 480. In that case, the total noise budget of 480 electrons rms implies an individual sample noise limit of (480)1/2 $\sim$ 22 electrons rms.  Although not directly applicable here, single channel devices with this noise level are commercially available.  It follows that low-noise performance might not lie outside the realm of possibility. This is enhanced by recent  
progress in CMOS devices.  The current design results for multi-pixel CMOS imagers suggest that a noise level of 0.39 electrons rms [REF] is possible.  

A possible solution for the readout pixel consists in a an array of 
tiny collection electrodes on the presenting surface of the dielectric; the grid is hidden---located on the other side of the dielectric surface.   The total capacitance at input of the exposed JFET electrode might be on the order of $\sim 1$~ fF.  The idea is that the resistivity of the dielectric is sufficiently high that charge on the surface has a very long lifetime relative to the arrival of charge from the TPC volume.  What this means is that if field lines exist that would cause charge to land on the surface, charge continues to arrive only as long as the field line exists in the gas. Eventually, all field lines in the dielectric terminate in surface charge. Field lines in the gas are deflected away.  In other words, the process equilibrates when a quasi-static distribution forms such that no field lines enter/leave the dielectric surface.   

In this scenario, free charge in the gas is guided toward the collection electrode. This process  requires time to reach electrostatic equilibrium after the drift field and grid potential are first turned on. The hidden grid should have some applied potential to help realize or maintain the desired field; this requires optimization. An external, movable, ionizing radiation source can be used to speed equilibration.  Because the equilibrium dielectric surface charge distribution is static, the imminent arrival of signal charge from events of interest does not induce changes in the distribution.  The hidden metallic grid dynamically supplies image charges as needed.

To summarize, the concept appears to provide small collecting electrode size, minimum capacitance, 
freedom from surface charge injection by the grid (but also good signal development due to the existence of the grid), and complete charge collection.  Of course, many questions remain such as: 
\begin{enumerate}
\item For a given stability requirement, what is the quantitative relationship between dielectric resistivity and current in the gas needed to sustain stability? 
\item How forgiving is this scenario against variations in dielectric resistivity or current?
\item In electrostatic equilibrium, if no field lines enter/leave the dielectric, does the grid need a potential different than the collection electrode? 
\end{enumerate}

\subsection{Backgrounds}
Like xenon, neither   
fluorine nor selenium have  long-lived energetic radioisotopes that would pose intrinsic background dangers. In practice the backgrounds sources to be dealt with are the same than in the case of \XE, that is electrons produced by \BI and \TL\  isotopes.  

On the other hand, the larger value of \Qbb\ for \SE\ implies that the photoelectric peak of \BI\ located at 2.445 MeV is not a significant source of background, which will be dominated by Compton electrons produced by gammas of higher energies. While a full background model for MIST needs yet to be computed, we believe that it is safe to assume that the background in the ROI will not be larger than for NEXT.   

\subsection{Detection: System Concept}

The detection system must collect both cation and anion images, both to the place the event in space, and to improve energy resolution.  The general scheme is shown in figure 3.  The detector is a cascade of symmetric TPCs, with readout planes operating at ±HV. Each TPC segment supports a drift length of about 1 meter.  The pressure vessel resembles a railroad tank car.  High temperature superconducting cable is wound around the exterior to provide a magnetic field.   At the ends of the pressure vessel, the TPC segments are either 1/2 length to permit operation at ground potential, or an insulating plate could replace the end segment.

The readout planes consist of dielectric planes with exposed JFETs, as elaborated above for design 3.  Considerable electronics is embedded behind the dielectric plane. To collect either the anion track image or cation track image, the readout plane is assumed to accept either polarity signal at run-time setup, realizing an identical design.  To meet or exceed noise goals, I assume that the challenges of Gigohm feed-back/reset resistor in the charge-sensitive amplifier can be met.  Subsequent signal shaping and processing includes a slow 2 - 5 kHz 12-bit ADC to permit accurate charge measurement.  The challenge of supporting mixed signal activity with miniscule electronic noise is clear.  Nearest-neighbor logic will be needed to include below-threshold ADC values needed to optimize offline integration of signal.  The power to operate each signal plane is transported through a multi-conductor HV connector, in the same way that conventional x-ray machines provide current at HV to operate the cathode filament. Digital communications and data transport is obviously done with fiber optics.  Altogether, the readout planes represent an interesting design opportunity. 

