\pdfoutput=1 % only if pdf/png/jpg images are used
\documentclass{JINST}

\usepackage{amsmath,amssymb}
\usepackage[numbers]{natbib}
%\usepackage{hyperref}
%\hypersetup{colorlinks, citecolor=blue, linkcolor=blue, filecolor=blue, urlcolor=red}

\title{Improved Background Rejection in Neutrinoless Double Beta Decay Experiments with Gaseous Xenon Detectors}

\author{A. Author$^a$,
B. Author$^a$\thanks{Corresponding author.}~
and C. Author$^b$\\
\llap{$^a$}Instituto de F\'isica Corpuscular (IFIC), CSIC \& Universitat de Val\`encia,\\ 
Calle Catedr\'atico Jos\'e Beltr\'an, 2, 46980 Paterna, Valencia, Spain\\
\llap{$^b$}Name of Institute,\\
  Address, Country\\
E-mail: \email{CorrespondingAuthor@email.com}}

\bibliographystyle{plainnat}

\abstract{We propose potential improvements in background rejection capabilities in neutrinoless double-beta ($0\nu\beta\beta$) decay experiments conducted with high-pressure xenon gas detectors capable of imaging electron tracks.  The improvements arise from the successful rejection of events consisting of a single energetic electron in favor of events consisting of two energetic electrons based on the analysis of tracks reconstructed in the detector.  The proposed method identifies single-electron tracks based on their direction of curvature in the presence of an external magnetic field.  This initial study shows that a potentially significant additional background rejection factor can be obtained with an acceptably small loss in signal efficiency using this strategy.}

\keywords{Neutrinoless double beta decay; Kalman filter; particle tracking}

\begin{document}

\section{Introduction}\label{sec:intro}
The observation of neutrinoless double-beta ($0\nu\beta\beta$) decay, in which two neutrons become two protons within a nucleus and two energetic electrons are emitted, would have several significant implications in fundamental physics.  First, the decay cannot occur unless the neutrino is a Majorana particle, that is, it is physically equivalent to its anti-particle, and the decay violates the conservation of total lepton number.\\

(Additional statements on $0\nu\beta\beta$ and NEXT?)

\section{Particle Tracks in Xenon in a Magnetic Field}\label{sec:magmotion}
A particle of charge $q$ moving at a velocity $\mathbf{v}$ in the presence of a magnetic field $\mathbf{B}$ is acted upon by a force

\begin{equation}
\mathbf{F} = q(\mathbf{v} \times \mathbf{B}).
\end{equation}

This force will cause an electron to execute helical motion in the magnetic field such that if the thumb of the right hand is positioned along the direction of the field line $\mathbf{B}$ the electron will spiral in the direction in which the fingers curl when closed around the field line (see figure \ref{fig_bfieldmotion}).  The frequency of rotation about the field line is known as the cyclotron frequency $\omega = qB/m$, where $B$ is the magnitude of the applied magnetic field $B = |\mathbf{B}|$ and $m$ is the mass of the charged particle.  
% radius of curvature $r = p_{T}/qB$

\begin{figure}[!htb]
	\centering
	\includegraphics[scale=0.48]{fig/bfield_motion.pdf}
	\caption{\label{fig_bfieldmotion}Motion of an electron in a magnetic field.  The electron exhibits helical motion around the field line as shown (entering the page at the $\bullet$ and exiting the page at the $\times$) regardless of the direction of the component of the electron's velocity along the direction of $\mathbf{B}$.}
\end{figure}

An energetic electron moving through high pressure xenon gas creates an ionization track as it deposits its energy.  In the presence of an applied external magnetic field, this track will be a helix with some alterations due to electron multiple scattering.  If the direction of the applied magnetic field is known, the curvature of the track can be calculated to determine whether the component of the electron velocity along the magnetic field is parallel or antiparallel to the field.  The curvature $\kappa$ can be calculated for track coordinates $(x,y,z)$ as

\begin{equation}\label{eqn_curv}
\kappa = \frac{(dx/dz)\cdot(d^2y/dz^2) - (dy/dz)\cdot(d^2x/dz^2)}{\Bigl[(dx/dz)^2 + (dy/dz)^2\Bigr]^{3/2}}.
\end{equation}

With this definition, an electron traveling in the direction of the magnetic field will spiral around the field lines with positive curvature, while an electron traveling opposite the direction of the magnetic field will spiral with negative curvature.  The curvature, however, will be of the opposite sign if the track orientation is not properly identified in the calculation (i.e., if $dz$ is of the wrong sign).  Thus, when calculating the curvature of a single-electron track, one would expect $\kappa > 0$ for $dz > 0$ and $\kappa < 0$ for $dz < 0$ given that $dz$ is always in the direction of the electron velocity.  However, for a $0\nu\beta\beta$ track, taking one of the extremes to be the beginning of the track and the other to be the end will lead to a calculation of $\kappa$ assuming the wrong track orientation for one of the two electrons, as the vertex at which the reaction occured is found somewhere on the interior of the track.  Therefore one expects to find $\kappa < 0$ for $dz < 0$ and $\kappa > 0$ for $dz > 0$ for a significant fraction of the track.  This difference in the behavior of the calculated curvature of reconstructed tracks will allow for the separation of single-electron and $0\nu\beta\beta$ events.

\section{Implementation}

\subsection{Track Preparation}\label{ssec:track}
The simulation-based study consisted of analysis of Monte Carlo datasets generated in a large virtual box of high pressure xenon gas using GEANT4.  For various configurations of gas pressure and magnetic field, 10000 events were generated consisting of single energetic electrons of kinetic energy equal to the Q-value of $0\nu\beta\beta$ in xenon gas ($Q_{\beta\beta} = 2.447$ MeV), and 10000 $0\nu\beta\beta$ events were generated consisting of two energetic electrons with total energy equal to $Q_{\beta\beta}$.  Each simulated track was recorded as a series of hits consisting of a location $(x,y,z)$ and a deposited energy $E$.

From this series of hits, a single continuous track was constructed by defining a main track as those hits produced directly by the one (or two in the case of $0\nu\beta\beta$ events) energetic electron(s) produced in the event.  Other hits produced by secondary ionization electrons were added to the main track if they were produced within 1 mm of at least one hit in the main track, and indirectly\footnote{Hits added indirectly were bunched into sub-tracks of hits lying within 1 mm of at least one other hit in the sub-track, and the energy-weighted centroid of the sub-track was calculated.  The total energy of the sub-track was then added to the hit closest to the location of the calculated centroid.} if they were produced within 2 cm of at least one hit in the main track.  Events containing hits located greater than 2 cm from all hits in the main track were discarded.  Note that the ordering of the hits was determined by Monte Carlo, however the orientation of the track was defined by constructing two ``blobs'' composed of hits within 2 cm of the first and last hits of the track, and setting as the ``initial'' end the one at which the constructed blob had less total energy.  The resulting list of hits was then smeared randomly in $(x,y)$ about their original values according to a gaussian distribution with sigma $\sigma_{s}$.  The hits were then ``sparsed,'' that is each group of $N_{s}$ hits was replaced with a single hit with $(x,y,z)$ location equal to the energy-weighted average of the constituent hit locations and energy equal to the sum of the constituent energies.  From this point on we will refer to the final list of ordered hits as the ``track.''
% In practice we will also have to determine the ordering in between hits!

\subsection{Determination of the Track Curvature}
The curvature at each hit in the track is calculated numerically by first separating the x-values, y-values, and z-values into their own arrays.  A low-pass FIR filter is designed to sufficiently smooth the track (additional details about the FIR filter) and applied to each of the arrays.  The derivatives $dx/dz$, $dy/dz$, $d^2x/dz^2$, and $d^2y/dz^2$ are calculated using the values in the array.  From these the curvature $\kappa$ is calculated at each point.  Since we do not have $x$ and $y$ as a function of $z$ but rather $x$, $y$, and $z$ as a function of hit number $n$, we can calculate the derivatives $x' \equiv dx/dn$, $y' \equiv dy/dn$, $z' \equiv dz/dn$ using the chain rule as $\frac{dx}{dz} = x'/z'$, and

\begin{equation}
\frac{d^2x}{dz^2} = \frac{x'' - z''(dx/dz)}{(z')^2}.
\end{equation}

The expressions for $dy/dz$ and $d^2y/dz^2$ can be obtained by replacing in the above $x \rightarrow y$.  Note that outliers may need to be removed from the resulting arrays of first and second derivatives due to points between which the z-coordinate changes very little.  To ensure more stable values of the derivatives, an outlier removal procedure is applied to all derivatives and second derivatives computed which consists of iteratively calculating the mean and variance $\sigma$ of each array, replacing any value that lies outside of $5\sigma$ of the mean value with the average of the two nearest values in the array, and continuing this procedure until the calculated variance $\sigma'^2$ is no longer less than the previous value of the variance $\sigma^2$. % (continue until $\sigma' < 0.99\sigma$).

The curvature calculated using each pair of points is then corrected as follows: if for the two points $z_2 < z_1$, that is $dz < 0$, the curvature is multiplied by -1 (see section \ref{sec:magmotion}).  Note that the outlier removal procedure described in step c) is also applied to the calculated curvature array.  Figure \ref{fig_trkcurv} shows the calculated curvature signs for a single-electron and double-beta event, after filtering with a lowpass filter.

\begin{figure}[!htb]
	\includegraphics[scale=0.48]{fig/plt_trkcurv_nmagse2_6.pdf}
	\includegraphics[scale=0.48]{fig/plt_trkcurv_nmagbb2_2.pdf}
	\caption{\label{fig_trkcurv}Calculated curvature sign at each point along the track for a single-electron event (left) and a double-beta event (right).}
\end{figure}

A curvature sign array is created consisting of values of either $+1$ or $-1$ depending on the sign of each value in the calculated curvature array.  The curvature asymmetry factor is defined as the average of the curvature sign array using elements in the first half of the track minus the average of the curvature sign array using elements in the second half of the track.

\begin{equation}\label{eqn_assym}
\phi_{C} = \frac{1}{N}\Biggl(\sum_{i=0}^{N/2-1}\mathrm{sgn}(\kappa_{i}) - \sum_{i=N/2}^{N}\mathrm{sgn}(\kappa_{i})\Biggr).
\end{equation}

\section{Results}
The asymmetry factor shown in equation \ref{eqn_assym} was calculated for each of 10000 single-electron and 10000 $0\nu\beta\beta$ tracks per configuration for all combinations of $P =$ 5, 10, and 15 atm and $B =$ 0.1, 0.3, 0.5, 0.7, and 1.0 T.  In all of these configurations, the hits were smeared in $(x,y)$ with $\sigma_{s} = 2$ mm and sparsed with $N_{s} = 2$.  A cut on the calculated asymmetry factors defining what events were considered to be candidate $0\nu\beta\beta$ tracks was varied, and in each case the fraction of single-electron (background) events rejected by the cut was determined along with the fraction of $0\nu\beta\beta$ (signal) events accepted.  An example of such an analysis is summarized in figure \ref{fig_svsbg}.

\begin{figure}[!htb]
	\includegraphics[scale=0.44]{fig/10atm_05T_scurv_diff_means.pdf}
	\includegraphics[scale=0.44]{fig/10atm_05T_sigvsb_all.pdf}
	\caption{\label{fig_svsbg}Curvature asymmetry factor for single-electron and $0\nu\beta\beta$ events (left) and resulting signal efficiency vs. background rejection curve produced by varying a cut on $\phi_{C}$ space (right).  These results were obtained for $P = 10$ atm and $B = 0.5$ T.}
\end{figure}

To demonstrate the performance of the method at the different configurations of gas pressure and magnetic field, we examine the signal efficiency obtained given 80\% background rejection.  The results are shown in figure \ref{fig_config}.

\begin{figure}[!htb]
	\centering
	\includegraphics[scale=0.43]{fig/eff_vs_b_80.pdf}
	\includegraphics[scale=0.43]{fig/eff_vs_b_90.pdf}
	\caption{\label{fig_config}Signal efficiency corresponding to 80\% background rejection (left) and 90\% background rejection (right) vs. magnitude of applied external magnetic field for different pressures.}
\end{figure}

%Show signal vs. background separation (should we be using different training and run datasets?) for different configurations and related plots; 80 pct and 90 pct signal efficiency factors vs. B-field for different pressures; final conclusions and recommendations

\section{Conclusions}
The application of an external magnetic field in a high-pressure xenon detector capable of particle track reconstruction with resolution of approximately 2 mm in $(x,y,z)$ could present an additional background rejection factor of 80\% with an acceptable loss of signal efficiency.\\

(Additional comments on how this will apply to NEXT)

%\newpage % Please avoid layout-changing commands if not strictly necessary

%\begin{figure}[tbp] % figures (and tables) should go top or bottom of
%                    % the page where they are first cited or in
%                    % subsequent pages
%\centering
%\includegraphics[width=.4\textwidth]{fig.png}
%\caption{Caption.}
%\label{fig:xxx}
%\end{figure}

%\begin{table}[tbp]
%\caption{Caption.}
%\label{tab:xxx}
%\smallskip
%\centering
%\begin{tabular}{|lc|}
%\hline
%a&b\\
%c&d\\
%\hline
%\end{tabular}
%\end{table}


\acknowledgments

This work was supported by the European Research Council under the Advanced Grant 339787-NEXT and the Ministerio de Econom\'{i}a y Competitividad of Spain under Grants CONSOLIDER-Ingenio 2010 CSD2008-0037 (CUP), FPA2009-13697-C04-04, FPA2009-13697-C04-01, FIS2012-37947-C04-01, FIS2012-37947-C04-02, FIS2012-37947-C04-03, and FIS2012-37947-C04-04.

\bibliography{nextb}

\end{document}
